\documentclass[a4paper,12pt]{article}

% Packages
\usepackage[utf8]{inputenc}
\usepackage[T1]{fontenc}
\usepackage[indonesian,english]{babel}
\usepackage{lmodern}
\usepackage{microtype}
\usepackage{setspace}
\usepackage{parskip}
\usepackage{geometry}
\usepackage{amsmath, amssymb}
\usepackage{graphicx}
\usepackage{hyperref}
\usepackage{enumitem}
\usepackage{fancyhdr}
\usepackage{titlesec}

% Include variables from the parent tex directory
% variables.tex
% Store all user variables for use in LaTeX and build.sh
\newcommand{\AuthorName}{Eez}
\newcommand{\GraduationYear}{2025}

% Define assignment title as a variable in the preamble
\newcommand{\AssignmentTitle}{Assignment 1: Quiz}

% Page setup
\geometry{margin=1in}
\onehalfspacing            % 1.5 line spacing

\sloppy

% Header & footer
\pagestyle{fancy}
\fancyhf{}
\fancyhead[L]{\AuthorName\ (\AuthorNIM)}
\fancyhead[R]{\AssignmentTitle\,\--\,Mata Kuliah KLM}
\fancyfoot[C]{\thepage}

% Section titles
\titleformat{\section}{\large\bfseries}{\thesection.}{1em}{}
\titleformat{name=\section,numberless}{\large\bfseries}{}{0pt}{}

% Custom title page
\renewcommand{\maketitle}{
  \begin{titlepage}
    \centering
    \includegraphics[width=0.2\textwidth]{../images/logo.png}\par\vspace{1cm}
    {\Huge\bfseries \AssignmentTitle}\par\vspace{0.5em}
    {\Large Mata Kuliah Kajian Literatur Musik}\par\vspace{2cm}
    {\large Nama: \AuthorName\\NIM: \AuthorNIM}\par\vspace{1cm}
    {\large 15 March 2025}\par
  \end{titlepage}
}

\begin{document}

\maketitle

\section*{Instruksi}
\begin{enumerate}
    \item Susun daftar kepustakaan beranotasi berdasarkan kata kunci yang diberikan.
    \item Format pengumpulan:
    \begin{itemize}
        \item Email ke: indrawan\_andre@isi.ac.id
        \item Subjek email: KLM\_QUIZ\_GNP 2025
        \item Nama file: KLM\_QUIZ\_[Nomor Presensi]\_[Nama].pdf
    \end{itemize}
\end{enumerate}

\section*{Tugas Kepustakaan Beranotasi}
Susunlah daftar kepustakaan beranotasi berdasarkan kata kunci berikut ini yang terkait dengan topik penelitian Anda. Setiap entri bibliografi harus disertai dengan anotasi yang menjelaskan isi dan relevansi sumber tersebut.

\subsection*{Judul Proyek Utama}
\ThesisTitle

\subsection*{Kata Kunci Penelitian}
\begin{itemize}
    \item Model Markov
    \item Pola ritme
    \item Komposisi algoritmik
    \item Musik komputasional
\end{itemize}

\vspace{1cm}

\section*{Daftar Kepustakaan Beranotasi}
\begin{enumerate}
    \item \textbf{Busemeyer, J., Zhang, Q., Balakrishnan, S., \& Wang, Z. (2020).} Application of Quantum—Markov Open System Models to Human Cognition and Decision. \textit{Entropy}, 22(9), 990. \href{https://doi.org/10.3390/e22090990}{DOI:10.3390/e22090990}.\par
    Artikel ini membahas penerapan model Markov dalam kognisi manusia dan pengambilan keputusan, mengintegrasikan aspek quantum ke dalam analisis Markov. Relevansi artikel ini terletak pada potensi penggunaannya dalam musik komputasional, di mana keputusan dan pola pembelajaran dapat diinterpretasikan melalui model matematis yang kompleks. Namun, tidak ada bukti konkret tentang hubungan langsung dengan musik komputasional.

    \item \textbf{Liu, H., Wang, K., \& Li, Y. (2020).} Hidden Markov Linear Regression Model and its Parameter Estimation. \textit{IEEE Access}, 8, 187037-187042. \href{https://doi.org/10.1109/access.2020.3030776}{DOI:10.1109/access.2020.3030776}.\par
    Penelitian ini memperkenalkan model regresi linier Markov tersembunyi dan metode estimasi parameternya. Model ini penting untuk memahami pola ritme dalam dataset musik, terutama dalam konteks pengolahan sinyal dan analisis temporal.

    \item \textbf{Ye, X., \& Wang, Y. (2014).} A Novel Method for Decoding Any High-Order Hidden Markov Model. \textit{Discrete Dynamics in Nature and Society}, 2014, 231704. \href{https://doi.org/10.1155/2014/231704}{DOI:10.1155/2014/231704}.\par
    Penelitian ini menawarkan metode baru untuk mendekode model Markov tersembunyi berorde tinggi yang dapat diaplikasikan dalam analisis pola dalam komposisi musik, berguna bagi pengembangan algoritma komposisi musik berbasis pola ritme kompleks.

    \item \textbf{Amini, A., et al. (2021).} hhsmm: An R Package for Hidden Hybrid Markov/Semi-Markov Models. arXiv:2109.12489.\par
    Artikel ini memperkenalkan paket R untuk model hybrid Markov/semi-Markov yang dapat digunakan untuk analisis musik, memungkinkan eksplorasi struktur dan probabilitas di balik komposisi musik.

    \item \textbf{Xia, Z., et al. (2018).} Bayesian Analysis for Hidden Markov Factor Analysis Models. IntechOpen. \href{https://doi.org/10.5772/intechopen.72837}{DOI:10.5772/intechopen.72837}.\par
    Artikel ini membahas analisis Bayesian untuk model analisis faktor Markov tersembunyi, menawarkan pendekatan efisien untuk menentukan pola dan elemen komposisi probabilistik dalam musik.

    \item \textbf{Verbeken, D., \& Guerry, P. (2021).} Discrete Time Hybrid Semi-Markov Models in Manpower Planning. \textit{Mathematics}, 9(14), 1681. \href{https://doi.org/10.3390/math9141681}{DOI:10.3390/math9141681}.\par
    Penelitian ini menjelaskan model semi-Markov yang diaplikasikan dalam perencanaan sumber daya. Prinsip dasar model semi-Markov dapat disesuaikan untuk menggambarkan perubahan dinamis pola ritme dalam komposisi musik.
\end{enumerate}

\vspace{1cm}

\end{document}