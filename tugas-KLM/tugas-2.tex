\documentclass[a4paper,12pt]{article}

% Packages
\usepackage[utf8]{inputenc}
\usepackage[T1]{fontenc}
\usepackage[indonesian]{babel}
\usepackage{lmodern}
\usepackage{microtype}
\usepackage{setspace}
\usepackage{parskip}
\usepackage{geometry}
\usepackage{amsmath, amssymb}
\usepackage{graphicx}
\usepackage{hyperref}
\usepackage{enumitem}
\usepackage{fancyhdr}
\usepackage{titlesec}

% Include variables from the parent tex directory
% variables.tex
% Store all user variables for use in LaTeX and build.sh
\newcommand{\AuthorName}{Eez}
\newcommand{\GraduationYear}{2025}

% Define assignment title as a variable in the preamble
\newcommand{\AssignmentTitle}{Assignment 2 : PRT}

% Page setup
\geometry{margin=1in}
\onehalfspacing            % 1.5 line spacing

\sloppy

% Header & footer
\pagestyle{fancy}
\fancyhf{}
\fancyhead[L]{\AuthorName\ (\AuthorNIM)}
\fancyhead[R]{\AssignmentTitle\,\--\,Mata Kuliah KLM}
\fancyfoot[C]{\thepage}

% Section titles
\titleformat{\section}{\large\bfseries}{\thesection.}{1em}{}
\titleformat{name=\section,numberless}{\large\bfseries}{}{0pt}{}

% Custom title page
\renewcommand{\maketitle}{
  \begin{titlepage}
    \centering
    \includegraphics[width=0.2\textwidth]{../images/logo.png}\par\vspace{1cm}
    {\Huge\bfseries \AssignmentTitle}\par\vspace{0.5em}
    {\Large Mata Kuliah Kajian Literatur Musik}\par\vspace{2cm}
    {\large Nama: \AuthorName\\NIM: \AuthorNIM}\par\vspace{1cm}
    {\large 30 April 2025}\par
  \end{titlepage}
}

\begin{document}

\maketitle

\section*{Instruksi}
\begin{enumerate}
    \item \textbf{Literature Review}
    \item Format pengumpulan:
    \begin{itemize}
        \item Email ke: dr.andre.indrawan@google.com
        \item Subjek email: KLM\_PRT\_GNP 2025
        \item Nama file: KLM\_PRT\_[Nomor Presensi]\_[Nama].pdf
    \end{itemize}
\end{enumerate}

\section*{Tugas Review Literatur}
Susunlah daftar kepustakaan beranotasi berdasarkan kata kunci berikut ini yang terkait dengan topik penelitian Anda. Setiap entri bibliografi harus disertai dengan anotasi yang menjelaskan isi dan relevansi sumber tersebut.

\subsection*{Judul Proyek Utama}
\ThesisTitle

\subsection*{Kata Kunci Penelitian}
\begin{itemize}
    \item Model Markov
    \item Pola ritme
    \item Komposisi algoritmik
    \item Musik komputasional
\end{itemize}

\vspace{1cm}

\vspace{1cm}

\end{document}