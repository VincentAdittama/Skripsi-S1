% statement.tex
% This is the statement page (HALAMAN PERNYATAAN) for the thesis

\begin{center}
    {\textbf{HALAMAN PERNYATAAN}}\par
    \vspace{0.4cm}
    {Judul Tugas Akhir}\par
    \vspace{0.4cm}
    
    {\textbf{\ThesisTitle}}\par
    \vspace{0.4cm}
\end{center}

{\setlength{\parindent}{0pt}
\setlength{\parskip}{0.5em}

{\noindent Dengan ini saya menyatakan bahwa skripsi ini benar-benar karya sendiri, sepanjang pengetahuan saya tidak terdapat karya atau pendapat yang ditulis atau diterbitkan oleh orang lain kecuali sebagai acuan atau kutipan dengan mengikuti tata penulisan karya ilmiah yang telah lazim.}

}

\vspace{1cm}

\begin{flushright}
    {Yogyakarta, \TanggalPernyataan}\par

    \vspace{2.5cm}

    {\AuthorName} \\
    {\AuthorNIM}
\end{flushright}
\newpage

% Motto page
\begin{center}
    {\textbf{MOTTO}}\par
    \vspace{0.4cm}

    \begin{redshadowbox}
    \centering
    {``Kesuksesan tidak datang dari apa yang diberikan oleh orang lain, tetapi datang dari keyakinan dan kerja keras kita sendiri.''}\par
    \vspace{0.5cm}
    
    -- Penulis\par
    \end{redshadowbox}
    
\end{center}

\newpage

% Kata Pengantar page
\begin{center}
    {\textbf{KATA PENGANTAR}}\par
    \vspace{0.1cm}
\end{center}

{\setlength{\parindent}{0.4in}
\setlength{\parskip}{0.5em}

{Alhamdulillah, puji dan syukur penulis panjatkan kehadirat Tuhan Yang Maha Esa atas segala rahmat dan hidayah-Nya, sehingga penulis dapat menyelesaikan Skripsi dengan judul ``\ThesisTitle''.}\par

{Skripsi ini disusun sebagai salah satu syarat untuk menyelesaikan studi dan memperoleh gelar Sarjana pada Program Studi \UniversityProgram, \UniversityFaculty, \UniversityName. Dalam penyelesaian skripsi ini, penulis telah mendapatkan banyak bantuan, bimbingan, dan dukungan dari berbagai pihak. Oleh karena itu, pada kesempatan ini penulis ingin menyampaikan ucapan terima kasih yang sebesar-besarnya kepada:}\par

\begin{enumerate}
    \raggedright % Apply ragged right alignment specifically for this list
    \item Bapak \PembimbingI, selaku Dosen Pembimbing Utama yang telah meluangkan waktu untuk memberikan bimbingan, arahan, dan masukan yang sangat berharga selama proses penyusunan skripsi ini.
    
    \item Ibu \PembimbingII, selaku Dosen Pembimbing Pendamping yang telah memberikan dukungan dan saran yang konstruktif untuk penyempurnaan skripsi ini.
    
    \item Bapak \KoordinatorProdi, selaku Koordinator Program Studi \UniversityProgram yang telah memberikan dukungan dan kemudahan dalam menyelesaikan studi.
    
    \item Seluruh Dosen \UniversityDepartment\ yang telah memberikan ilmu pengetahuan dan pengalaman yang bermanfaat selama masa perkuliahan.
    
    \item Staf dan karyawan \UniversityFaculty\ yang telah membantu dalam proses administrasi selama masa studi.
    
    \item Kedua orang tua tercinta dan seluruh keluarga yang selalu memberikan doa, dukungan moral, dan material yang tak terhingga.
    
    \item Teman-teman mahasiswa \UniversityDepartment\ angkatan 2019 yang telah memberikan dukungan, bantuan, dan kerja sama selama masa perkuliahan.
    
    \item Semua pihak yang tidak dapat disebutkan satu per satu yang telah membantu dalam penyelesaian skripsi ini.
\end{enumerate}

Penulis menyadari bahwa skripsi ini masih jauh dari sempurna. Oleh karena itu, kritik dan saran yang membangun sangat diharapkan untuk penyempurnaan skripsi ini. Semoga skripsi ini dapat bermanfaat bagi pengembangan ilmu pengetahuan dan semua pihak yang membutuhkan.

}

\vspace{1cm}

\begin{flushright}
    {Yogyakarta, \TanggalPengesahan}\par
    {Penulis}\par
    \vspace{2cm}
    {\AuthorName}
\end{flushright}

\newpage

